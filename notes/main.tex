
\documentclass{article}

\usepackage{hyperref}
\usepackage{parskip}
\usepackage{fullpage}
\usepackage{amsmath}
\usepackage{graphviz}
\usepackage{tabu}

\usepackage[backend=biber,autocite=footnote,
			bibstyle=authortitle,citestyle=verbose-inote]{biblatex}
\addbibresource{main.bib}
\setlength\bibitemsep{1em}

\usepackage{listings}
\lstset{numbers=left,
		float=htpb,
		language=Matlab,
		basicstyle=\footnotesize,
		captionpos=b,
		showspaces=false,
		showstringspaces=false,
		xleftmargin=0.3in}

\raggedright

\usepackage{tikz}
\usetikzlibrary{calc,arrows,positioning,shadows}
%
% control system tikz objects
\tikzset{shadow/.style={drop shadow,fill=white}}
\tikzset{sum/.style={circle,draw,very thick,shadow,minimum size=5mm}}
\tikzset{sumnp/.style={sum,
				label={{above left,xshift=0.0mm,yshift=-1.0mm}:$+$},
				label={{below right,xshift=-1.0mm,yshift=0.0mm}:$-$}} }
\tikzset{sumpp/.style={sum,
				label={{above left,xshift=0.0mm,yshift=-1.0mm}:$+$},
				label={{below right,xshift=-1.0mm,yshift=0.0mm}:$+$}} }
\tikzset{sumpn/.style={sum,
				label={{below left,xshift=0.0mm,yshift=1.0mm}:$+$},
				label={{above right,xshift=-1.0mm,yshift=0.0mm}:$-$}} }
\tikzset{sumxpxp/.style={sum,
				label={{above left,xshift=0.0mm,yshift=-1.0mm}:$+$},
				label={{above right,xshift=-0.0mm,yshift=-1.0mm}:$+$}} }
\tikzset{gain/.style={rectangle,draw,very thick,inner sep=2mm,shadow}}

\usepackage[]{pdfpages}
% \sincludepdf command sets common options for \includepdf
\newcommand{\sincludepdf}[2][]{
	\includepdf[scale=0.8,pagecommand={},#1]{#2}
}

\usepackage{titlesec}
\setcounter{secnumdepth}{4}

% For fourth level headers, use \paragraph
\titleformat{\paragraph}
{\normalfont\normalsize\bfseries}{}{1em}{}
\titlespacing*{\paragraph}
{0pt}{3.25ex plus 1ex minus .2ex}{1.5ex plus .2ex}

\begin{document}

% {{{ title page

\vspace*{0.5in}

\centerline{\LARGE \textbf{Control System Design}}
\vspace{0.2in}

\begin{center}
\begin{tabular}{c}
Jeremiah Mahler \texttt{<jmmahler@gmail.com>} \\
\today
\end{tabular}
\end{center}

\thispagestyle{empty}
\vfill
\pagebreak

% }}}

\tableofcontents

\clearpage

\section{Introduction}

This document is a collection of notes on control system design.
The author collected these while taking several control systems
classes at California State University Chico taught by Dr. Adel Ghandakly.
The first was an Introduction to Control Systems (EECE 482) and
the second was Computer Control of Dynamic Systems (EECE 682).
The former focused on continuous systems whereas the latter focused on
digital systems suitable for implentation in a computer.

Matlab code is included for most examples.
They have been tested with Octave\autocite{octave},
an open source Matlab equivalent.

% {{{ Common Plants/Systems
\clearpage
\section{Common Plants/Systems}

\nocite{ogata1995discrete}
\nocite{franklin1998digital}

Plants denoted by $G(s)$, systems by $H(s)$ and controllers by $D(s)$.

{\tabulinesep=1.5mm
\begin{tabu}{|c|l|}
\hline
$\displaystyle G(s) = \frac{1}{s(s + 1)}$ & Pole Placement (Section \ref{sec:pp}), ZOH (Section \ref{sec:zoh}), Direct Design ($K$) (Section \ref{sec:dd}) \\
$\displaystyle G(s) = \frac{1}{s(s + 7)}$ & \\
$\displaystyle G(s) = \frac{a}{s(s + a)}$ & Mapping (Section \ref{sec:mapzest}), ZOH (Section \ref{sec:zoh}) \\
$\displaystyle G(s) = \frac{1}{(s + 1)^2}$ & \\
$\displaystyle G(s) = \frac{1}{s(s + 0.4)}$ & PID (Section \ref{sec:pid}), ZOH (Section \ref{sec:zoh}) \\
$\displaystyle G(s) = \dfrac{1}{s^2}$ & ZOH (Section \ref{sec:zoh}), Direct Design (Section \ref{sec:dd}), double integrator \\
$\displaystyle G(s) = \frac{10}{(s + 0.1)(s + 0.2)}$ & \\
$\displaystyle G(z) = 0.0484 \frac{z + 0.9672}{(z - 1)(z - 0.9048)}$ & antenna model\autocite[Pg. 261]{franklin1998digital} \\
\hline
$\displaystyle H(s) = \frac{10s + 1}{1 + s}$ & \\
$\displaystyle H(s) = \frac{20}{s^2 + 4s + 20}$ &  System From Specs (Section \ref{sfs:ex1}), Pole Placement (\ref{sec:pp}) \\
$\displaystyle H(s) = \frac{10}{s^2 + s + 1}$ & \\
$\displaystyle \frac{B_m}{A_m} = \frac{0.62z - 0.3}{z^2 - 1.2z + 0.52}$ & Model Matching (Section \ref{sec:mm})\autocite[Pg. 532]{ogata1995discrete} \\

\hline
$\displaystyle D(s) = \frac{149s + 880}{s + 47}$ & Pole Placement (Section \ref{sec:pp}) \\
$\displaystyle D(s) = \frac{(s + 7)}{1.5 + 2.5 + s^2}$ & \\
$\displaystyle D(s) = \frac{20s^2 + 20s}{s^2 + 4s}$ & Pole Placement (Section \ref{sec:pp}) \\
\hline
\end{tabu}}

%\begin{align}
%	G(s) &= \frac{1}{s^2} & \mbox{(Double Integrator)}\\
%	G(s) &= \frac{1}{(s + 1)^2} \\
%	H(s) &= \frac{1 + 10s}{1 + s} \\
%	G(s) &= \frac{a}{s(s + a)} \\
%	G(s) &= \frac{1}{s(s + 1)} \\
%	G(s) &= \frac{1}{s(s + 7)} \\
%	D(s) &= \frac{(s + 7)}{1.5 + 2.5 + s^2} \\
%	G(s) &= \frac{1}{s(s + 0.4)} \\
%	H(s) &= \frac{10}{s^2 + s + 1} \\
%	G(s) &= \frac{10}{(s + 0.1)(s + 0.2)} \\
%	H(s) &= \frac{20}{s^2 + 4s + 20} & \mbox{(from specs: $Tr = 0.5$, $\%OS = 25$)} \\
%	G(z) &= 0.0484 \frac{z + 0.9672}{(z - 1)(z - 0.9048)}\label{eq:antenna}
%\end{align}
%Equation \ref{eq:antenna} describes an antenna
%model\autocite[Pg. 261]{franklin1998digital}.

% }}}

\sincludepdf[pages={1},
		pagecommand=\section{Laplace Transform}\subsection*{Example 1}
	]{scan/11211301.pdf}

% {{{ Systems From Specifications
\section{Systems From Specifications}

\begin{align}
	H(s) &= \frac{\omega_n^2}{s^2 + 2 \zeta \omega_n s + \omega_n^2}
\end{align}

There are different ways to find $\omega_n$ and $\zeta$.
The following specs for the rise time ($T_r$), over shoot ($\%OS$),
and setting time ($T_s$) are one example.
\begin{align*}
	T_r &= \frac{2.22}{\omega_n} \\
	\%OS &= \left( 1 - \frac{\zeta}{0.6} \right) \cdot 100 \\
	T_s &= \frac{4}{\zeta \omega_n}
\end{align*}

\sincludepdf[pages={3},
			pagecommand=\subsection*{Example 1}\label{sfs:ex1},
		]{scan/11221301.pdf}

The step response of this system is shown in Figure \ref{fig:spec0525_plot}.
It can be seen that the specs are met when a unit step is applied.
Interestingly, these specs are satisfied even if the input signal is
scaled and shifted as shown by Figure \ref{fig:spec0525s_plot}.

\begin{figure}[h!]
\begin{center}
\includegraphics[scale=0.5]{../system_specs/spec0525_plot}
\end{center}
\caption{Step response for system with $Tr = 0.5$ and $\%OS = 25$.}
\label{fig:spec0525_plot}
\end{figure}

\begin{figure}[h!]
\begin{center}
\includegraphics[scale=0.6]{../system_specs/spec0525s_plot}
\end{center}
\caption{Shifted and scaled step response with $Tr = 0.5$
and $\%OS = 25$ (same as previous).}
\label{fig:spec0525s_plot}
\end{figure}

% }}}

\clearpage
\sincludepdf[pages={3},
		pagecommand=\section{Final Value Theorem}
		]{scan/11241301.pdf}

% {{{ Pole Placement, Polynomial Matching
% Example 1
\sincludepdf[pages={2},
	pagecommand=\section{Pole Placement (S-Domain)}\label{sec:pp}\subsection*{Example 1}\label{sec:pp:ex1}
	]{scan/11231301.pdf}
\sincludepdf[pages={3}]{scan/11231301.pdf}

% calculation of step response
\sincludepdf[pages={5}]{scan/11241301.pdf}

The step response of this system is show in Figure \ref{fig:pps2s1_plot}.
The Matlab code used to produce this plot is shown
in Listing \ref{lst:pps2s1_plot}.
It can be seen that it agrees with calculated steady state error.

\begin{figure}[h!]
\begin{center}
\includegraphics[scale=0.6]{../pole_placement_cont/pps2s1_plot}
\end{center}
\caption{Step response with no control and with controller
built using Pole Placement.}
\label{fig:pps2s1_plot}
\end{figure}

\clearpage
\lstinputlisting[
	caption={Matlab script to plot step response of Pole Place controller.},
	label=lst:pps2s1_plot,
]{../pole_placement_cont/pps2s1_plot.m}

% calculation of ramp response
\sincludepdf[pages={4}]{scan/11241301.pdf}

The ramp response of this system is show in Figure \ref{fig:pps2s1r_plot}.
The Matlab code used to produce this plot is shown
in Listing \ref{lst:pps2s1r_plot}.
It can be seen that it agrees with calculated steady state error.

\begin{figure}[h!]
\begin{center}
\includegraphics[scale=0.6]{../pole_placement_cont/pps2s1r_plot}
\end{center}
\caption{Ramp response of controller built using Pole Placement.}
\label{fig:pps2s1r_plot}
\end{figure}

\lstinputlisting[
	caption={Matlab script to plot ramp response of Pole Place controller.},
	label=lst:pps2s1r_plot,
]{../pole_placement_cont/pps2s1r_plot.m}

% {{{ Example 2
\clearpage
\sincludepdf[pages={4},
			pagecommand=\subsection*{Example 2}
	]{scan/11231301.pdf}

The step response of this system is show in Figure \ref{fig:pps2s1a_plot}.
The Matlab code used to produce this plot is shown
in Listing \ref{lst:pps2s1a_plot}.

\begin{figure}[h!]
\begin{center}
\includegraphics[scale=0.6]{../pole_placement_cont/pps2s1a_plot}
\end{center}
\caption{Step response of controller found using algebraic methods.}
\label{fig:pps2s1a_plot}
\end{figure}

\lstinputlisting[
	caption={Matlab script to plot step response of controller
		found using algebraic methods.},
	label=lst:pps2s1a_plot,
]{../pole_placement_cont/pps2s1a_plot.m}

% calculation of ramp response
\sincludepdf[pages={6}]{scan/11241301.pdf}

The ramp response of this system is show in Figure \ref{fig:pps2s1ar_plot}.
The Matlab code used to produce this plot is shown
in Listing \ref{lst:pps2s1ar_plot}.
It can be seen that the calculated steady state error for a ramp
response agrees with the plot.

\begin{figure}[h!]
\begin{center}
\includegraphics[scale=0.6]{../pole_placement_cont/pps2s1ar_plot}
\end{center}
\caption{Ramp response of controller found using algebraic methods.}
\label{fig:pps2s1ar_plot}
\end{figure}

\lstinputlisting[
	caption={Matlab script to plot ramp response of controller
		found using algebraic methods.},
	label=lst:pps2s1ar_plot,
]{../pole_placement_cont/pps2s1ar_plot.m}

% }}}

% }}}

% {{{ old example [DISABLED]
% old example, not significantly different than others
%\sincludepdf[pages={4},
%			pagecommand=\section{Pole Placement, Polynomial Matching}\subsection*{Example 1}
%		]{scan/11221301.pdf}
%
%The response of this system is show in Figure \ref{fig:pps2s4_plot}.
%The Matlab code used to produce this plot is shown
%in Listing \ref{lst:pps2s4_plot}.
%
%\begin{figure}[h!]
%\begin{center}
%\includegraphics[scale=0.6]{../pole_placement_cont/pps2s4_plot}
%\end{center}
%\caption{Step response of controller built using Pole Placement.}
%\label{fig:pps2s4_plot}
%\end{figure}
%
%\lstinputlisting[
%	caption={Matlab script to plot response of controller built using
%	Pole Placement.},
%	label=lst:pps2s4_plot,
%]{../pole_placement_cont/pps2s4_plot.m}
%
%\clearpage
%\begin{figure}[h!]
%\begin{center}
%\includegraphics[scale=0.6]{../pole_placement_cont/pps2s4r_plot}
%\end{center}
%\caption{Ramp response of controller built using Pole Placement.}
%\label{fig:pps2s4r_plot}
%\end{figure}
%
%\lstinputlisting[
%	caption={Matlab script to plot ramp response of controller built
%	using Pole Placement.},
%	label=lst:pps2s4r_plot,
%]{../pole_placement_cont/pps2s4r_plot.m}
% }}}

\clearpage
\section{Pole Placement, Diophantine}

\section{Steady State Performance/Error}

See Section \ref{sec:pp} for examples with steady state error ($\varepsilon_{ss}$).

% {{{ PID Controller Design

\sincludepdf[pages={5},
	pagecommand=\section{PID Controller Design}\label{sec:pid}\subsection*{Example 1}
		]{scan/11221301.pdf}

% calculation of step response
\sincludepdf[pages={7}]{scan/11241301.pdf}

The step response of this system is show in Figure \ref{fig:pids2s4_plot}.
The Matlab code used to produce this plot is shown
in Listing \ref{lst:pids2s4_plot}.

\begin{figure}[h!]
\begin{center}
\includegraphics[scale=0.6]{../pid_cont/pids2s4_plot}
\end{center}
\caption{Step response of PID controller.}
\label{fig:pids2s4_plot}
\end{figure}

\lstinputlisting[
	caption={Matlab script to plot response of PID controller.},
	label=lst:pids2s4_plot,
]{../pid_cont/pids2s4_plot.m}

% calculation of ramp response
\sincludepdf[pages={8}]{scan/11241301.pdf}

The ramp response of this system is show in Figure \ref{fig:pids2s4r_plot}.
The Matlab code used to produce this plot is shown
in Listing \ref{lst:pids2s4r_plot}.
Notice that the response does not agree with the calculated error.

\begin{figure}[h!]
\begin{center}
\includegraphics[scale=0.6]{../pid_cont/pids2s4r_plot}
\end{center}
\caption{Ramp response of PID controller.}
\label{fig:pids2s4r_plot}
\end{figure}

\lstinputlisting[
	caption={Matlab script to plot ramp response of PID controller.},
	label=lst:pids2s4r_plot,
]{../pid_cont/pids2s4r_plot.m}

% }}}

% {{{ State Space to Transfer Function (S-Domain)
\sincludepdf[pages={2},
			pagecommand=\section{State Space to Transfer Function (S-Domain)}\subsection*{Example 1}
	]{scan/11211301.pdf}
\sincludepdf[pages={3},
		pagecommand={}
	]{scan/11211301.pdf}

% }}}

% {{{ Transfer Function (S-Domain) to State Space

\clearpage
\sincludepdf[pages={4},
			pagecommand=\section{Transfer Function (S-Domain) to State Space}\subsection*{Example 1}
		]{scan/11211301.pdf}
\sincludepdf[pages={5}]{scan/11211301.pdf}

% }}}

% {{{ Z-transform
\section{Z-transform}

\begin{align}
	X(z) = \sum_{n = -\infty}^{\infty} x(n) z^{-n}
\end{align}

\sincludepdf[pages={6},
		pagecommand=\subsection*{Example 1}
	]{scan/11211301.pdf}

\sincludepdf[pages={7},
		pagecommand=\section{Geometric Series}
	]{scan/11211301.pdf}
% }}}

% {{{ Algorithm (Z-Domain) to Difference Equation
\section{Algorithm (Z-Domain) to Difference Equation}

\sincludepdf[pages={11},
			pagecommand=\subsection*{Example 1}
		]{scan/11211301.pdf}

\sincludepdf[pages={12},
			pagecommand=\subsection*{Example 2}
		]{scan/11211301.pdf}

\sincludepdf[pages={13},
			pagecommand=\subsection*{Example 3}
		]{scan/11211301.pdf}
% }}}

% {{{ Zero Order Hold
\clearpage
\section{Zero Order Hold}
\label{sec:zoh}

A Zero Order Hold converts a S-domain system to the Z-domain.
It is effectively the same as putting an D/A converter before
the continuous system.
In fact this is exactly what is done in Simulink (Figure \ref{fig:simulinkzoh}).

The Zero Order Hold is typically used for $G(s)$, the plant/model.
Mapping operations such as Forward, Backward, etc
(Section \label{sec:mapping}) are not typically used.

% {{{ ZOH figure
% TODO - update to new style
\begin{figure}
\begin{center}
\tikzstyle{block}=[draw,minimum size=2.4em]
\tikzstyle{sum}=[draw,circle,minimum size=1.2em]
%\tikzstyle{init} = [pin edge={to-,thin,black}]
\begin{tikzpicture}[node distance=2.0cm,auto,>=latex']

	% place all the block, points, relative to each other
	\node [coordinate,node distance=2cm] (u) {};
	\node [block,right of=u] (da) {$D/A$};
	\node [coordinate,right of=da,node distance=2cm] (d) {};
	\node [block,right of=da] (g) {$G(s)$};
	\node [coordinate,right of=g,node distance=2cm] (y) {};

	\node [block,below of=d] (gz) {$G(z)$};
	\node [coordinate,below of=u,node distance=2cm] (uz) {};
	\node [coordinate,below of=y,node distance=2cm] (yz) {};

	% connect all the nodes and points
	\path[->] (u) edge node {} (da);
	\path[->] (da) edge node {} (g);
	\path[->] (g) edge node {} (y);

	\path[->] (uz) edge node {} (gz);
	\path[->] (gz) edge node {} (yz);

\end{tikzpicture}
\end{center}

\caption{A Zero Order Hold in Simulink created by preceding
a $G(s)$ system with a D/A converter to produce a $G(z)$ system.}
\label{fig:simulinkzoh}
\end{figure}
% }}}

\begin{align}
	G(z) &= (1 - z^{-1}) Z \left[ \frac{G(s)}{s} \right]
\end{align}

All of these examples can be verified in Matlab using
code as shown below.
There may be small differences due to round off errors.

\begin{lstlisting}
% Matlab
Bs = [1];
As = [1 0];
Gs = tf(Bs, As);
Gz = c2d(Gs, T, 'ZOH');
\end{lstlisting}

\sincludepdf[pages={8},
			pagecommand=\subsection*{Example 1}
	]{scan/11211301.pdf}

\sincludepdf[pages={6},
			pagecommand=\subsection*{Example 2}
		]{scan/11221301.pdf}
\sincludepdf[pages={7}]{scan/11221301.pdf}

\sincludepdf[pages={1},
			pagecommand=\subsection*{Example 3}
		]{scan/11221301.pdf}
\sincludepdf[pages={2}]{scan/11221301.pdf}

\sincludepdf[pages={1},
			pagecommand=\subsection*{Example 4}
		]{scan/11241301.pdf}
\sincludepdf[pages={2}]{scan/11241301.pdf}

\sincludepdf[pages={1},
			pagecommand=\subsection*{Example 5}
		]{scan/11251301.pdf}
\sincludepdf[pages={2}]{scan/11251301.pdf}

% }}}

% {{{ Mapping (S-domain to Z-domain)
\section{Mapping (S-Domain to Z-Domain)}
\label{sec:mapping}

When designing by discrete equivalents the design is performed
in the continuous domain and then converted to the discrete domain.
This is in contrast to Direct Design (Section \ref{sec:dd}) where
the designed is performed in the discrete domain.

This method is straightforward however it doesn't always result in
a stable system.

\sincludepdf[pages={10},
		pagecommand=\subsection{Mapping: $z=e^{sT}$}\label{sec:mapzest}\subsubsection*{Example 1}
		]{scan/11221301.pdf}

\sincludepdf[pages={1},
			pagecommand=\subsubsection*{Example 2}
	]{scan/11231301.pdf}

\sincludepdf[pages={9},
			pagecommand=\subsection{Mapping: Forward, Backward, Trapezoid}\subsubsection*{Example 1}
	]{scan/11211301.pdf}

\sincludepdf[pages={10},
			pagecommand=\subsubsection*{Example 2}
	]{scan/11211301.pdf}

\sincludepdf[pages={8},
			pagecommand=\subsubsection*{Example 3}
		]{scan/11221301.pdf}

\sincludepdf[pages={9},
			pagecommand=\subsubsection*{Example 4}
		]{scan/11221301.pdf}

\subsection{Digital PID, Ghandakly's Method}

% }}}

% {{{ Direct Design Method of Ragazzini
\section{Direct Design Method of Ragazzini}

One method for finding $D(z)$, if $H(z)$ and $G(z)$ are given,
is to simply solve for $D(z)$\autocite[Pg. 265]{franklin1998digital}.

\begin{align}
	H(z) &= \frac{D(z)G(z)}{1 + D(z)G(z)} \notag \\
	D(z) &= \frac{1}{G(z)} \frac{H(z)}{1 - H(z)}
\end{align}

% }}}

% {{{ $K_v$ Direct Design Method
\section{$K_v$ Direct Design Method}

This method is characterized by placing a limit on the error
produced from a ramp input ($K_v$).
The error due to a step input ($K_p$) will also be zero.

Alternative names for this method are the Direct Method
of Ragazzini\autocite[Pg. 264]{franklin1998digital} and
the Analytical Design Method\autocite[Pg. 242]{ogata1995discrete}.
% }}}

% {{{ Diophantine Equation
\section{Diophantine Equation}

The Diophantine Equation is used to find a solution to a system
if it is in a very specific form (Equation \ref{eq:diophantine}).
For more information refer to Ogata\autocite[Pg. 525]{ogata1995discrete}
where this method is called the ``Polynomial Equations Approach''.

\begin{align}
	\alpha(z)A(z) + \beta(z)B(z) &= D \label{eq:diophantine}
\end{align}

Where $D$ is the characteristic polynomial.
Typically, $A(z)$ and $B(z)$ are known and $\alpha(z)$ and $\beta(z)$
are to be found.
Each element must have a specific order as shown below.
The order ($n$) will correspond to the order of the
Sylvester Matrix (Section \ref{sec:sylvester}).

\[
\begin{array}{ccccccc}
	\alpha(z) &A(z) &+ &\beta(z) &B(z) &= &D \\
	  (n-1)   &(n)  &  & (n-1)  &(n) & &(2n-1)
\end{array}
\]

When $A(z)$ and $B(z)$ are known $\alpha(z)$ and $\beta(z)$
can be found using Equation \ref{eq:diopmat}.

\begin{align}
	M &= E^{-1} D \label{eq:diopmat} \\
%	M &:
%	\left[
%	\begin{array}{l}
%		\alpha \\
%		\beta
%	\end{array}
%	\right] \notag \\
	\left[
	\begin{array}{l}
		\alpha \\
		\beta
	\end{array}
	\right] &= E^{-1} D \notag
\end{align}

\subsection{Sylvester Matrix}
\label{sec:sylvester}

Second Order
\[
E=
\begin{bmatrix}
	a_2 & 0 & b_2 & 0 \\
	a_1 & a_2 & b_1 & b_2 \\
	a_0 & a_1 & b_0 & b_1 \\
	0 & a_0 & 0 & b_0 \\
\end{bmatrix}
\]

Third Order
\[
E=
\begin{bmatrix}
	a_3 & 0 & 0 & b_3 & 0 & 0 \\
	a_2 & a_3 & 0 & b_2 & b_3 & 0 \\
	a_1 & a_2 & a_3 & b_1 & b_2 & b_3 \\
	a_0 & a_1 & a_2 & b_0 & b_1 & b_2 \\
	0 & a_0 & a_1 & 0 & b_0 & b_1 \\
	0 & 0 & a_0 & 0 & 0 & b_0 \\
\end{bmatrix}
\]

% }}}

% {{{ Direct Design ($K$), Diophantine
\clearpage
\section{Direct Design ($K$), Diophantine}
\label{sec:dd}

% {{{ dd figure
\begin{figure}[h!]

\begin{center}
\tikzstyle{block}=[draw,minimum size=2.4em]
\tikzstyle{sum}=[draw,circle,minimum size=1.2em]
%\tikzstyle{init} = [pin edge={to-,thin,black}]
% TODO - update to new style
\begin{tikzpicture}[node distance=2.0cm,auto,>=latex']

	% place all the block, points, relative to each other
	\node [coordinate,node distance=2cm] (u) {u};
	\node [block,right of=u] (k) {$K$};
	\node [sum,right of=k] (s) {};
	\node [coordinate,below of=s] (ds) {ds};
	\node [block,right of=ds] (d) {$D(s)$};
	\node [block,right of=s] (g) {$G(s)$};
	\node [coordinate,right of=g] (gdy) {};
	\node [coordinate,right of=gdy,node distance=2cm] (y) {y};
	\node [coordinate,right of=d] (gd) {gd};

	% add signs to sum
	\node [left of=s,node distance=0cm,xshift=-3mm,yshift=3mm] {$+$};
	\node [left of=s,node distance=0cm,xshift=-3mm,yshift=-4mm] {$-$};

	% add labels to G(s) and D(s)
	\node [right of=g,node distance=0cm,xshift=8mm,yshift=6mm] {$\dfrac{B}{A}$};
	\node [right of=d,node distance=0cm,xshift=8mm,yshift=6mm] {$\dfrac{\beta}{\alpha}$};

	% connect all the nodes and points
	\path[->] (u) edge node {$u$} (k);
	\path[->] (k) edge node {} (s);
	\path[->] (ds) edge node {} (s);
	\path[-]  (d) edge node {} (ds);
	\path[->] (s) edge node {$\epsilon$} (g);
	\path[-] (g) edge node {} (gdy);
	\path[->] (gdy) edge node {$y$} (y);
	\path[-] (gdy) edge node {} (gd);
	\path[->] (gd) edge node {} (d);

\end{tikzpicture}
\end{center}

\caption{Direct Design system with $K$ as a scaling input, $G$ is
the plant and $D$ (${\beta}/{\alpha}$) is the controller.}
\label{fig:dd}
\end{figure}
% }}}

The general structure for a Direct Design system is shown in
Figure \ref{fig:dd}.
This is different than previous designs in that the controller ($D$)
is in the feedback loop and there is a gain ($K$) on the input.

A controller is found by first converting the plant ($G(s)$) in to discrete
form using a Zero Order Hold (Section \ref{sec:zoh}).
Then the poles are placed according to roots given by the designer.
Roots may need to be added to correct the order so that the Diophantine
Equation can be used to find $\alpha$ and $\beta$.
Examples of this process are included below.
For a detailed examination refer to Ogata\autocite[Pg. 517]{ogata1995discrete}.

\begin{align}
	H &= \frac{y}{u} \notag \\
	H &= K \frac{G}{1 + D G} \\
	  &= K \frac{B \alpha}{A \alpha + B \beta}
\end{align}

% {{{ Example 1
\sincludepdf[pages=1,
			pagecommand=\subsection*{Example 1}
	]{scan/11211302.pdf}
\sincludepdf[pages=2-5]{scan/11211302.pdf}

Listing \ref{lst:dd1s2_init} shows the Matlab code used to perform
these calculations.
Listing \ref{lst:dd1s2_plot} shows the Matlab code to plot the step response
as shown in Figure \ref{fig:dd1s2_plot}.
The \verb+sylvester+ function is given in Appendix \ref{app:sylvester}.

\lstinputlisting[
	caption={Matlab script to find Direct Design of double integrator.},
	label=lst:dd1s2_init,
]{../direct_design/dd1s2_init.m}

\clearpage
\lstinputlisting[
	caption={Matlab script to plot the response of the double integrator.},
	label=lst:dd1s2_plot,
]{../direct_design/dd1s2_plot.m}

\begin{figure}
\begin{center}
\includegraphics[scale=0.6]{../direct_design/dd1s2_plot}
\end{center}
\caption{Step response of double integrator for controller built
using Direct Design. Uses a time step of 0.2 seconds.}
\label{fig:dd1s2_plot}
\end{figure}

\clearpage
Listing \ref{lst:dd1s2r_plot} shows the Matlab code to plot the ramp response
as shown in Figure \ref{fig:dd1s2r_plot}.

\lstinputlisting[
	caption={Matlab script to plot the ramp response of the double integrator.},
	label=lst:dd1s2r_plot,
]{../direct_design/dd1s2r_plot.m}

\begin{figure}
\begin{center}
\includegraphics[scale=0.6]{../direct_design/dd1s2r_plot}
\end{center}
\caption{Ramp response of double integrator for controller built
using Direct Design. Uses a time step of 0.2 seconds.}
\label{fig:dd1s2r_plot}
\end{figure}

% }}}

% {{{ Example 2
\sincludepdf[pages=3,
			pagecommand=\subsection*{Example 2}
	]{scan/11251301.pdf}
\sincludepdf[pages=4-6]{scan/11251301.pdf}

Listing \ref{lst:dd1s2s_init} shows the Matlab code used to perform
these calculations.
Listing \ref{lst:dd1s2s_plot} shows the Matlab code to plot the step response
as shown in Figure \ref{fig:dd1s2s_plot}.
The \verb+sylvester+ function is given in Appendix \ref{app:sylvester}.

\lstinputlisting[
	caption={Matlab script to perform the Direct Design calculations.},
	label=lst:dd1s2s_init,
]{../direct_design/dd1s2s_init.m}

\clearpage
\lstinputlisting[
	caption={Matlab script to plot the step response.},
	label=lst:dd1s2s_plot,
]{../direct_design/dd1s2s_plot.m}

\begin{figure}
\begin{center}
\includegraphics[scale=0.6]{../direct_design/dd1s2s_plot}
\end{center}
\caption{Step response of controller built
using Direct Design. Uses a time step of 0.2 seconds.}
\label{fig:dd1s2s_plot}
\end{figure}

\clearpage
Listing \ref{lst:dd1s2sr_plot} shows the Matlab code to plot the ramp response
as shown in Figure \ref{fig:dd1s2sr_plot}.

\lstinputlisting[
	caption={Matlab script to plot the ramp response.},
	label=lst:dd1s2sr_plot,
]{../direct_design/dd1s2sr_plot.m}

\begin{figure}
\begin{center}
\includegraphics[scale=0.6]{../direct_design/dd1s2sr_plot}
\end{center}
\caption{Ramp response of controller built
using Direct Design. Uses a time step of 0.2 seconds.}
\label{fig:dd1s2sr_plot}
\end{figure}

% }}}

% }}}

% {{{ Model Matching
\clearpage
\section{Model Matching ($G_{model}$)}
\label{sec:mm}

% {{{ mm figure [disabled]
% This is the figure was given by Dr. Ghandakly but it
% doesn't appear to be correct compared to Ogata Pg. 533.
%\begin{figure}[h!]
%
%\begin{center}
%\tikzstyle{block}=[draw,minimum size=2.4em]
%\tikzstyle{sum}=[draw,circle,minimum size=1.2em]
%%\tikzstyle{init} = [pin edge={to-,thin,black}]
%\begin{tikzpicture}[node distance=2.0cm,auto,>=latex']
%
%	% place all the block, points, relative to each other
%	\node [coordinate,node distance=2cm] (u) {};
%	\node [block,right of=u] (k) {$G_{model}$};
%	\node [sum,right of=k] (s) {};
%	\node [coordinate,below of=s] (ds) {ds};
%	\node [block,right of=ds] (d) {$D(z)$};
%	\node [block,right of=s] (g) {$G(z)$};
%	\node [coordinate,right of=g] (gdy) {};
%	\node [coordinate,right of=gdy,node distance=2cm] (y) {};
%	\node [coordinate,right of=d] (gd) {};
%
%	% add signs to sum
%	\node [left of=s,node distance=0cm,xshift=-3mm,yshift=3mm] {$+$};
%	\node [left of=s,node distance=0cm,xshift=-3mm,yshift=-4mm] {$-$};
%
%	% add labels to G(s) and D(s)
%	\node [right of=g,node distance=0cm,xshift=8mm,yshift=6mm] {$\dfrac{B}{A}$};
%	\node [right of=d,node distance=0cm,xshift=8mm,yshift=6mm] {$\dfrac{\beta}{\alpha}$};
%
%	% connect all the nodes and points
%	\path[->] (u) edge node {$R$} (k);
%	\path[->] (k) edge node {} (s);
%	\path[->] (ds) edge node {} (s);
%	\path[-]  (d) edge node {} (ds);
%	\path[->] (s) edge node {$\epsilon$} (g);
%	\path[-] (g) edge node {} (gdy);
%	\path[->] (gdy) edge node {$y$} (y);
%	\path[-] (gdy) edge node {} (gd);
%	\path[->] (gd) edge node {} (d);
%
%\end{tikzpicture}
%\end{center}
%
%\caption{Model Matching system with $G_{model}$ as input. $G$ is
%the plant and $D$ is the controller.}
%\label{fig:mm}
%\end{figure}
% }}}

Suppose the perfect system was created and then the plant changed.
How could the system response be reproduced identically with this new plant?
One solution is to use Model Matching\footnote{If this method was called
System Matching it would make more sense because it matches $H(z)$.}.

Given a model system ($G_{model}$) a controller is found for the
given plant ($G(z)$).

\begin{align}
	\frac{Y(z)}{R(z)} &= G_{model} = \frac{B_m(z)}{A_m(z)}
\end{align}

For more details refer to Ogata\autocite[Pg. 532]{ogata1995discrete}.

% {{{ Example 1
\sincludepdf[pages=1,
			pagecommand=\subsection*{Example 1}
	]{scan/11261301.pdf}
\sincludepdf[pages=2-4]{scan/11261301.pdf}

\clearpage
Listing \ref{lst:mm3679_init} shows the Matlab code used to perform
these calculations.
Listing \ref{lst:mm3679_plot} shows the Matlab code to plot the step response
as shown in Figure \ref{fig:mm3679_plot}.
The \verb+sylvester+ function is given in Appendix \ref{app:sylvester}.

\lstinputlisting[
	caption={Matlab script to perform Model Matching for Example 1.},
	label=lst:mm3679_init,
]{../model_matching/mm3679_init.m}

\clearpage
\lstinputlisting[
	caption={Matlab script to plot the step response of Example 1.},
	label=lst:mm3679_plot,
]{../model_matching/mm3679_plot.m}

\begin{figure}
\begin{center}
\includegraphics[scale=0.6]{../model_matching/mm3679_plot}
\end{center}
\caption{Step response of controller built using Model Matching.
Notice that $G_{model}$ is identical to the whole system as expected.}
\label{fig:mm3679_plot}
\end{figure}

\clearpage
Listing \ref{lst:mm3679r_plot} shows the Matlab code to plot the ramp response
as shown in Figure \ref{fig:mm3679r_plot}.

\lstinputlisting[
	caption={Matlab script to plot the ramp response of Example 1.},
	label=lst:mm3679r_plot,
]{../model_matching/mm3679r_plot.m}

\begin{figure}
\begin{center}
\includegraphics[scale=0.6]{../model_matching/mm3679r_plot}
\end{center}
\caption{Ramp response of controller built using Model Matching.}
\label{fig:mm3679r_plot}
\end{figure}

% }}}

% }}}

\clearpage
\section{Poll Placement with Ackerman's Formula}

% {{{ Control Law, Full Order, No Prediction
\clearpage
\section{Control Law, Full Order, No Predicition}

A Control Law system behaves as a regulator as shown in
Figure \ref{fig:clff1}.
In this case it is assumed that all outputs are observable and
so are fed back.
Later examples will investigate partial feedback with estimation.
Ackermann's formula is used to find $K$.

% {{{ fig:clff1
\begin{figure}[hpb!]
\begin{center}

\begin{tikzpicture}[>=triangle 60,
	node distance=14mm, auto]

	\node[gain]  	(Gz)	[] 	{$\begin{matrix}
									\dot{x} = Ax + Bu \\
										 y = Cx + Du
								 \end{matrix}$};
	\node[gain] 	(K)	[below=of Gz] 	{$-K$};

	 \draw [->]
	 		(Gz.east) -- ++(15mm,0)
			-| ++(0,-10mm)
			|- (K.east);

	 \draw [<-]
	 		(Gz.west) -- ++(-15mm,0)
			-| ++(0,-10mm)
			|- (K.west);

\end{tikzpicture}

\end{center}

\caption{Control Law system with full feedback.
It behaves as a regulator since not input is available.}
\label{fig:clff1}
\end{figure}
% }}}

\subsection{Example 1}

For this example the calculation of $K$ is shown in Listing \ref{lst:clff1}.
The \verb+myacker+ function is a re-implementation of the Matlab \verb+acker+
function and is given in Appendix \ref{app:myacker}.
Figure \ref{fig:cl_fonp02} shows the response of this system.

\lstinputlisting[
	caption={Matlab script to calculate $K$ using Ackermann's formula.},
	label=lst:clff1,
]{../control_law/cl_fonp02.m}

\begin{figure}[h!]
\begin{center}
\includegraphics[scale=0.7]{../control_law/cl_fonp02}
\end{center}
\caption{Response of Control Law designed regulator system
for two different sets of roots.}
\label{fig:cl_fonp02}
\end{figure}

% }}}

% {{{ Control Law, Full Order, Predictor Estimator
\clearpage
\section{Control Law, Full Order, Predicitor Estimator}

% TODO
%A Control Law system behaves as a regulator as shown in
%Figure \ref{fig:clff1}.
%In this case it is assumed that all outputs are observable and
%so are fed back.
%Later examples will investigate partial feedback with estimation.
%Ackermann's formula is used to find $K$.

% {{{ fig:clff1  TODO
%\begin{figure}[hpb!]
%\begin{center}
%
%\begin{tikzpicture}[>=triangle 60,
%	node distance=14mm, auto]
%
%	\node[gain]  	(Gz)	[] 	{$\begin{matrix}
%									\dot{x} = Ax + Bu \\
%										 y = Cx + Du
%								 \end{matrix}$};
%	\node[gain] 	(K)	[below=of Gz] 	{$-K$};
%
%	 \draw [->]
%	 		(Gz.east) -- ++(15mm,0)
%			-| ++(0,-10mm)
%			|- (K.east);
%
%	 \draw [<-]
%	 		(Gz.west) -- ++(-15mm,0)
%			-| ++(0,-10mm)
%			|- (K.west);
%
%\end{tikzpicture}
%
%\end{center}
%
%\caption{Control Law system with full feedback.
%It behaves as a regulator since not input is available.}
%\label{fig:clff1}
%\end{figure}
% }}}

\subsection{Example 1}

% TODO
%For this example the calculation of $K$ is shown in Listing \ref{lst:clpr1}.
%The \verb+myacker+ function is a re-implementation of the Matlab \verb+acker+
%function and is given in Appendix \ref{app:myacker}.
%Figure \ref{fig:cl_ex82} shows the response of this system.

\lstinputlisting[
	caption={Matlab script to calculate $K$ and
		estimate $L$ using Ackermann's formula.},
	label=lst:clpr1,
]{../control_law/cl_fope01.m}

\begin{figure}[h!]
\begin{center}
\includegraphics[scale=0.7]{../control_law/cl_fope01}
\end{center}
\caption{Response of Control Law system designed with a full
order predictor estimator compared to a system with no prediction.
Initial $x$ has one in first position with zeros elsewhere.}
\label{fig:cl_fope01}
\end{figure}

% }}}

\clearpage
\section{Control Law, Reduced Order Estimator}

Partial feedback with estimation.

\clearpage
\printbibliography[heading=bibintoc]

\clearpage
\appendix

\section{Sylvester Matrix Generation in Matlab}
\label{app:sylvester}

\lstinputlisting[
	caption={Matlab function to calculate a Sylvester Matrix.},
	label=lst:sylvester,
]{../lib/sylvester.m}

\section{Ackermann's Formula in Matlab}
\label{app:myacker}

\lstinputlisting[
	caption={Matlab function to calculate Ackermann's Formula.
	Designed as an example of how Matlab's acker could
	be implemented.},
	label=lst:myacker,
]{../lib/myacker.m}

\section{Quick Reference}

\section{Table of Laplace and Z-transforms}

\end{document}
