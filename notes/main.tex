
\documentclass{article}

\usepackage{parskip}
\usepackage{fullpage}
\usepackage{amsmath}
\usepackage{graphviz}

\usepackage[backend=biber,autocite=footnote,
			bibstyle=authortitle,citestyle=verbose-inote]{biblatex}
\addbibresource{main.bib}
\setlength\bibitemsep{1em}

\usepackage{listings}
\lstset{numbers=left,
		float=htpb,
		language=Matlab,
		basicstyle=\footnotesize,
		captionpos=b,
		xleftmargin=0.3in}

\raggedright

\usepackage{tikz}
\usetikzlibrary{arrows}

\usepackage[]{pdfpages}
% \sincludepdf command sets common options for \includepdf
\newcommand{\sincludepdf}[2][]{
	\includepdf[scale=0.8,pagecommand={},#1]{#2}
}

\usepackage{titlesec}
\setcounter{secnumdepth}{4}

% For fourth level headers, use \paragraph
\titleformat{\paragraph}
{\normalfont\normalsize\bfseries}{\theparagraph}{1em}{}
\titlespacing*{\paragraph}
{0pt}{3.25ex plus 1ex minus .2ex}{1.5ex plus .2ex}

\begin{document}

% {{{ title page

\vspace*{0.5in}

\centerline{\LARGE \textbf{Notes on Control System Design}}
\vspace{0.2in}

\begin{center}
\begin{tabular}{c}
Jeremiah Mahler \texttt{<jmmahler@gmail.com>} \\
\today
\end{tabular}
\end{center}

\thispagestyle{empty}
\vfill
\pagebreak

% }}}

\tableofcontents

\clearpage

\section{Introduction}

This document is a collection of notes on control system design.
The author collected this information while taking two control systems
classes at California State University Chico taught by Dr. Adel Ghandakly.
The first was an introduction to control systems (EECE 482) and
the second was an advanced control systems engineering (EECE 682).
The former focused on continuous systems whereas the latter was on
discrete.

Many of the examples are hand written and place in line.
It would be nice to have these properly formatted in \LaTeX but
for the time being hand written is more expedient.

% {{{ Common Plants/Systems
\section{Common Plants/Systems}

\nocite{ogata1995discrete}
\nocite{franklin1998digital}

Plants denoted by $G(s)$ and systems by $H(s)$.

\begin{align*}
	G(s) &= \frac{1}{s^2} & \mbox{(Double Integrator)}\\
	G(s) &= \frac{1}{(s + 1)^2} \\
	H(s) &= \frac{1 + 10s}{1 + s} \\
	G(s) &= \frac{a}{s(s + a)} \\
	G(s) &= \frac{1}{s(s + 0.4)} \\
	G(s) &= \frac{10}{s^2 + s + 1} \\
	G(s) &= \frac{10}{(s + 0.1)(s + 0.2)}
\end{align*}

% }}}

\clearpage
\section{Continuous}

\subsection{Laplace Transform}

\sincludepdf[pages={1}]{scan/11211301.pdf}

\subsection{Second Order System From Specifications}

\begin{align}
	H(s) &= \frac{\omega_n^2}{s^2 + 2 \zeta \omega_n s + \omega_n^2}
\end{align}

There are different ways to find $\omega_n$ and $\zeta$.
The following specs for the rise time ($T_r$), over shoot ($\%OS$),
and setting time ($T_s$) are one example.
\begin{align*}
	T_r &= \frac{2.22}{\omega_n} \\
	\%OS &= \left( 1 - \frac{\zeta}{0.6} \right) \cdot 100 \\
	T_s &= \frac{4}{\zeta \omega_n}
\end{align*}

\sincludepdf[pages={3},
			pagecommand=\subsubsection*{Example 1}
		]{scan/11221301.pdf}

\sincludepdf[pages={4},
			pagecommand=\subsection{Pole Placement}\subsubsection*{Example 1}
		]{scan/11221301.pdf}

\subsection{Steady State Performance}

\subsection{Pole Placement, Diophantine}

% {{{ PID Controller Design
\subsection{PID Controller Design}

\sincludepdf[pages={5},
			pagecommand=\subsubsection*{Example 1}
		]{scan/11221301.pdf}

% }}}

\subsection{State Space to Transfer Function (S-domain)}

\sincludepdf[pages={2-3}]{scan/11211301.pdf}

\subsection{Transfer Function (S-domain) to State Space}

\sincludepdf[pages={4},
			pagecommand=\subsubsection*{Example 1}
		]{scan/11211301.pdf}
\sincludepdf[pages={5}]{scan/11211301.pdf}

\section{Digital}

\subsection{Z-transform}

\begin{align}
	X(z) = \sum_{n = -\infty}^{\infty} x(n) z^{-n}
\end{align}

\sincludepdf[pages={6},
		pagecommand=\subsubsection*{Example 1}
	]{scan/11211301.pdf}

\sincludepdf[pages={7},
		pagecommand=\subsection{Geometric Series}
	]{scan/11211301.pdf}

\subsection{S-domain to Z-domain}
%\subsection{Design by Z-Domain Equivalents}

When designing by discrete equivalents the design is performed
in the continuous domain and then converted to the discrete domain.
This is in contrast to direct design (Section \ref{sec:dd}) where
it is designed in the discrete domain without using the continuous domain.

To check stability, entire system must be in Z-domain.



\sincludepdf[pages={10},
			pagecommand=\subsubsection{Mapping: $z=e^{sT}$}\subsubsection*{Example 1}
		]{scan/11221301.pdf}

\sincludepdf[pages={9},
			pagecommand=\subsubsection{Mapping: Forward, Backward, Trapezoid}\paragraph{Example 1}
	]{scan/11211301.pdf}

\sincludepdf[pages={10},
			pagecommand=\paragraph{Example 2}
	]{scan/11211301.pdf}

\sincludepdf[pages={8},
			pagecommand=\subsubsection*{Example 3}
		]{scan/11221301.pdf}

\sincludepdf[pages={9},
			pagecommand=\subsubsection*{Example 4}
		]{scan/11221301.pdf}

\subsubsection{Digital PID, Ghandakly's Method}

% {{{ Algorithm (Z-Domain) to Difference Equation
\subsection{Algorithm (Z-Domain) to Difference Equation}

\sincludepdf[pages={11},
			pagecommand=\subsubsection*{Example 1}
		]{scan/11211301.pdf}

\sincludepdf[pages={12},
			pagecommand=\subsubsection*{Example 2}
		]{scan/11211301.pdf}

\sincludepdf[pages={13},
			pagecommand=\subsubsection*{Example 3}
		]{scan/11211301.pdf}
% }}}

% {{{ Diophantine Equation
\subsection{Diophantine Equation}

The Diophantine Equation is used to find a solution to a system
if it is in a very specific form (Equation \ref{eq:diophantine}).
For more information refer to Ogata\autocite[Pg. 525]{ogata1995discrete}
where this method is called the ``Polynomial Equations Approach''.

\begin{align}
	\alpha(z)A(z) + \beta(z)B(z) &= D \label{eq:diophantine}
\end{align}

Where $D$ is the characteristic polynomial.
Typically, $A(z)$ and $B(z)$ are known and $\alpha(z)$ and $\beta(z)$
are to be found.
Each element must have a specific order as shown below.
The order ($n$) will correspond to the order of the
Sylvester Matrix (Section \ref{sec:sylvester}).

\[
\begin{array}{ccccccc}
	\alpha(z) &A(z) &+ &\beta(z) &B(z) &= &D \\
	  (n-1)   &(n)  &  & (n-1)  &(n) & &(2n-1)
\end{array}
\]

When $A(z)$ and $B(z)$ are known $\alpha(z)$ and $\beta(z)$
can be found using Equation \ref{eq:diopmat}.

\begin{align}
	M &= E^{-1} D \label{eq:diopmat} \\
%	M &:
%	\left[
%	\begin{array}{l}
%		\alpha \\
%		\beta
%	\end{array}
%	\right] \notag \\
	\left[
	\begin{array}{l}
		\alpha \\
		\beta
	\end{array}
	\right] &= E^{-1} D \notag
\end{align}

\subsubsection{Sylvester Matrix}
\label{sec:sylvester}

Second Order
\[
E=
\begin{bmatrix}
	a_2 & 0 & b_2 & 0 \\
	a_1 & a_2 & b_1 & b_2 \\
	a_0 & a_1 & b_0 & b_1 \\
	0 & a_0 & 0 & b_0 \\
\end{bmatrix}
\]

Third Order
\[
E=
\begin{bmatrix}
	a_3 & 0 & 0 & b_3 & 0 & 0 \\
	a_2 & a_3 & 0 & b_2 & b_3 & 0 \\
	a_1 & a_2 & a_3 & b_1 & b_2 & b_3 \\
	a_0 & a_1 & a_2 & b_0 & b_1 & b_2 \\
	0 & a_0 & a_1 & 0 & b_0 & b_1 \\
	0 & 0 & a_0 & 0 & 0 & b_0 \\
\end{bmatrix}
\]

% }}}

% {{{ Zero Order Hold
\subsubsection{Zero Order Hold}

A Zero Order Hold converts a S-domain system to the Z-domain.
It is effectively the same as putting an D/A converter before
the continuous system.
In fact this is exactly what is done in Simulink (Figure \ref{fig:simulinkzoh}).

\begin{figure}
(TODO)
\caption{A Zero Order Hold in Simulink created by preceding
a $G(s)$ system with a D/A converter to produce a $G(z)$ system.}
\label{fig:simulinkzoh}
\end{figure}

\begin{align}
	G(z) &= (1 - z^{-1}) Z \left[ \frac{G(s)}{s} \right]
\end{align}

\begin{lstlisting}
% Matlab
Gs = tf(Bs, As);
Gz = c2d(Gs, T, 'ZOH');
\end{lstlisting}

\sincludepdf[pages={8},
			pagecommand=\paragraph{Example 1}
	]{scan/11211301.pdf}

\sincludepdf[pages={6},
			pagecommand=\subsubsection*{Example 2}
		]{scan/11221301.pdf}
\sincludepdf[pages={7}]{scan/11221301.pdf}

\sincludepdf[pages={1},
			pagecommand=\subsubsection*{Example 3}
		]{scan/11221301.pdf}
\sincludepdf[pages={2}]{scan/11221301.pdf}


% }}}

% {{{ Direct Design
\clearpage
\subsection{Direct Design ($K$)}
\label{sec:dd}

For Direct Design the system is in the form shown in Figure \ref{fig:dd}.
For more information refer to Ogata\autocite[Pg. 517]{ogata1995discrete}.

% {{{ dd figure
\begin{figure}

\begin{center}
\tikzstyle{block}=[draw,minimum size=2.4em]
\tikzstyle{sum}=[draw,circle,minimum size=1.2em]
%\tikzstyle{init} = [pin edge={to-,thin,black}]
\begin{tikzpicture}[node distance=2.0cm,auto,>=latex']

	% place all the block, points, relative to each other
	\node [coordinate,node distance=2cm] (u) {u};
	\node [block,right of=u] (k) {$K$};
	\node [sum,right of=k] (s) {};
	\node [coordinate,below of=s] (ds) {ds};
	\node [block,right of=ds] (d) {$D(s)$};
	\node [block,right of=s] (g) {$G(s)$};
	\node [coordinate,right of=g] (gdy) {};
	\node [coordinate,right of=gdy,node distance=2cm] (y) {y};
	\node [coordinate,right of=d] (gd) {gd};

	% add signs to sum
	\node [left of=s,node distance=0cm,xshift=-3mm,yshift=3mm] {$+$};
	\node [left of=s,node distance=0cm,xshift=-3mm,yshift=-4mm] {$-$};

	% add labels to G(s) and D(s)
	\node [right of=g,node distance=0cm,xshift=8mm,yshift=6mm] {$\dfrac{B}{A}$};
	\node [right of=d,node distance=0cm,xshift=8mm,yshift=6mm] {$\dfrac{\beta}{\alpha}$};

	% connect all the nodes and points
	\path[->] (u) edge node {$u$} (k);
	\path[->] (k) edge node {} (s);
	\path[->] (ds) edge node {} (s);
	\path[-]  (d) edge node {} (ds);
	\path[->] (s) edge node {$\epsilon$} (g);
	\path[-] (g) edge node {} (gdy);
	\path[->] (gdy) edge node {$y$} (y);
	\path[-] (gdy) edge node {} (gd);
	\path[->] (gd) edge node {} (d);

\end{tikzpicture}
\end{center}

\caption{Direct Design system wit $K$ as a scaling input, $G$ is
the plant and $D$ (${\beta}/{\alpha}$) is the controller.}
\label{fig:dd}
\end{figure}
% }}}

\clearpage

\sincludepdf[
	pages=1,
	pagecommand=\subsubsection*{Example: $\dfrac{1}{s^2}$}
	]{scan/11211302.pdf}
\sincludepdf[pages=2-5]{scan/11211302.pdf}

Listing \ref{lst:dd1s2_init} shows the Matlab code used to perform
these calculations.
Listing \ref{lst:dd1s2_plot} shows the Matlab code to plot the response.
Figure \ref{fig:dd1s2_plot} shows the response.

\lstinputlisting[
	caption={Matlab script to find Direct Design of double integrator.},
	label=lst:dd1s2_init,
]{../direct_design/dd1s2_init.m}

\lstinputlisting[
	caption={Matlab script to plot the response of the double integrator.},
	label=lst:dd1s2_plot,
]{../direct_design/dd1s2_plot.m}

\begin{figure}
\begin{center}
\includegraphics[scale=0.6]{../direct_design/dd1s2_plot}
\end{center}
\caption{Response of double integrator for controller built
using Direct Design.}
\label{fig:dd1s2_plot}
\end{figure}

% }}}

% {{{ Model Matching
\clearpage
\subsection{Model Matching ($G_{model}$)}
\label{sec:mm}

Controller design using Model Matching is similar to Direct Design
(Section \ref{sec:dd}) except that instead of a gain ($K$) there is
another transfer function ($G_{model}$).
For more information refer to Ogata\autocite[Pg. 532]{ogata1995discrete}.

\begin{figure}

\begin{center}
\tikzstyle{block}=[draw,minimum size=2.4em]
\tikzstyle{sum}=[draw,circle,minimum size=1.2em]
%\tikzstyle{init} = [pin edge={to-,thin,black}]
\begin{tikzpicture}[node distance=2.0cm,auto,>=latex']

	% place all the block, points, relative to each other
	\node [coordinate,node distance=2cm] (u) {u};
	\node [block,right of=u] (k) {$G_{model}$};
	\node [sum,right of=k] (s) {};
	\node [coordinate,below of=s] (ds) {ds};
	\node [block,right of=ds] (d) {$D(z)$};
	\node [block,right of=s] (g) {$G(z)$};
	\node [coordinate,right of=g] (gdy) {};
	\node [coordinate,right of=gdy,node distance=2cm] (y) {y};
	\node [coordinate,right of=d] (gd) {gd};

	% add signs to sum
	\node [left of=s,node distance=0cm,xshift=-3mm,yshift=3mm] {$+$};
	\node [left of=s,node distance=0cm,xshift=-3mm,yshift=-4mm] {$-$};

	% add labels to G(s) and D(s)
	\node [right of=g,node distance=0cm,xshift=8mm,yshift=6mm] {$\dfrac{B}{A}$};
	\node [right of=d,node distance=0cm,xshift=8mm,yshift=6mm] {$\dfrac{\beta}{\alpha}$};

	% connect all the nodes and points
	\path[->] (u) edge node {$u$} (k);
	\path[->] (k) edge node {} (s);
	\path[->] (ds) edge node {} (s);
	\path[-]  (d) edge node {} (ds);
	\path[->] (s) edge node {$\epsilon$} (g);
	\path[-] (g) edge node {} (gdy);
	\path[->] (gdy) edge node {$y$} (y);
	\path[-] (gdy) edge node {} (gd);
	\path[->] (gd) edge node {} (d);

\end{tikzpicture}
\end{center}

\caption{Model Matching system with $G_{model}$ as input. $G$ is
the plant and $D$ is the controller.}
\label{fig:mm}
\end{figure}
% }}}

\subsection{Poll Placement with Ackerman's Formula}

\subsection{State Space, Full Feedback}

Full feedback with no estimation.

\subsection{State Space, Full Order "predictor" Estimator}

Full feedback but some are estimated.

\subsection{State Space, Reduced Order Estimator}

Partial feedback with estimated.

\clearpage
\printbibliography[heading=bibintoc]

\clearpage
\appendix

\section{Sylvester Matrix in Matlab}

\lstinputlisting[
	caption={Matlab function to calculate a Sylvester Matrix.},
	label=lst:sylvester,
]{../direct_design/sylvester.m}


\section{Quick Reference}

\subsection{Table of Laplace and Z-transforms}

\end{document}
